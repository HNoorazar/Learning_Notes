\section{Missing Values}
\label{sec:Missing_Values}
The issue of missing values is a real one
that is present in any dataset. It is inevitable and impossible to avoid it.

Rubin~\citep{rubin1976inference} divides the missing data mechanisms
into three categories; missing at random (MAR) missing completely at random (MCAR),
and missing not at random (MNAR).
\begin{description}
\item [MCAR] Missing completely at random (bad luck), e.g. Server dies.
This is the safest case. 
drop rows.\\
{\bf Test} Litte's MCAR test. If MCAR is rejected then it is either MAR or MNAR.\\
{\bf Imputation:} drop rows, mean, mode, KNN, regression, MICE.

\item [MAR] Missing At Random: missing value depends on observed variables and does not depend on the missing value itsef. you can explain why data is missing by other columns.
eg: old people don't do medical tests.

{\bf Imputation:} MICE, maximum likelihood, joint modeling, use tree-based methods, category for missing values.

methods such as multi imputation, max-likelihood rely on this assumption
\item [MNAR] They are missing because of the unobserved value itself.
e.g. high income people do not report their income.
Requires modeling missingness process explicitly. we cannot detect MNAR from data. you don't see what caused missingness. $P(\text{missing} | \text{value})$-- and not $p(\text{value} | \text{missing})$

MAR and MNAR are statistically indistinguishable.
We can some times reject MAR, but not confirm MNAR.
Standard imputation can be biased. 

{\bf Imputation:} Explicitly model missingness, Pattern-mixture models, 
\end{description}


In this section we focus our attention to the problem of
missing values. Let us start with defining the simplest methods
of dealing with such issue.

\begin{deff}{}
Listwise deletion is a method for handling missing data where an entire record 
is excluded from analysis if any single value is missing.
\end{deff}{}

\begin{deff}{}
Pairwise deletion occurs when the statistical procedure uses cases 
that contain some missing data. The procedure cannot include a 
particular variable when it has a missing value, but it can still use 
the case when analyzing other variables with non-missing values. 
A case may contain 3 variables: $x_1$, $x_2$, and $x_3$. A case may 
have a missing value for $x_1$, but this does not prevent some 
statistical procedures from using the same case to analyze variables 
$x_2$ and $x_3$. 
\end{deff}{}
Pairwise deletion allows you to use more of your 
data. However, each computed statistic may be based on a 
different subset of cases. 
This can be problematic. For example, a correlation matrix computed 
using pairwise deletion may not be positive semidefinite. 
That is, it may have negative eigenvalues, which can create problems 
for various statistical analyses. This can occur because when correlations 
are computed using different cases, the resulting patterns can be 
ones that are impossible to produce with complete data.

It is important to understand that in the vast majority of cases, 
an important assumption to using either of these techniques is 
that your data is missing completely at random (MCAR).  
In other words, the researcher needs to support that the probability 
of missing data on their dependent variable is unrelated to other 
independent variables as well as the dependent variable itself.

``Newer and principled methods, such as the multiple-imputation (MI) 
method, the full information maximum likelihood (FIML) method, 
and the expectation-maximization (EM) method, take into consideration 
the conditions under which missing data occurred and provide better 
estimates for parameters than either list deletion or pairwise deletion.
Principled missing data methods do not replace a missing value directly; 
they combine available information from the observed data with statistical 
assumptions in order to estimate the population parameters and/or the 
missing data mechanism statistically''~\citep{dong2013principled}.


Before choosing an approach for imputing missing data
one has to examine missing data mechanism and pattern of
missing data.\\

MICE (different from MI?) acts in the following way
\begin{itemize}
\item model $\text{income} = f(\text{age}, \text{education})$
\item model $\text{education} = g(\text{age}, \text{income})$
\item model $\text{age} = h(\text{income}, \text{education})$
\item repeat
\end{itemize}
