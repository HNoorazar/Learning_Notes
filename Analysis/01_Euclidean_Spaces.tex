\chapter{Fleming}
%%%%%%
%%%%%%  01_Exercises
%%%%%%
\section{Euclidean Spaces}
\label{Euclidean-Spaces}

\subsection{The real number system}

\textbf{Problems}
\begin{enumerate}
\item Find $\sup S$ and $\inf S$ for the following sets
and state whether $\sup S$ and $\inf S$ are elements of $S$.
\begin{enumerate}[label=(\alph*)]
\item $S = \{x: x^2-3x+2 < 0 \}$.

\sol The roots of $x^2-3x+2 = 0$ are 1 and 2.
The function is concave up. So, $\inf S = 1$ and $\sup S = 2$
and obviously they do not belong to $S$.

\item $S = \{x: x^3 +x^2 - 2x \le 2 \}$.

\sol Let $f(x) = x^3 +x^2 - 2x - 2$. 
We are looking for values $x$ where $f(x) \le 0$.
We can rewrite $f(x) = (x^2 - 2) (x + 1)$. Roots of
$f(x)$ are $-1, \pm\sqrt{2}$.
From the figure of the function $\sup S = \max S = \sqrt{2}$.
And similarly, or by a close look at the property given
in definition of $S$ we can guess $\inf S = - \infty$.

\item $S = \{ sin x + cos x : x \in [0, \pi] \}$.

\sol Obviously for $x \in [0, \pi]$ we have
$ -1 \le sin x +cos x \le \sqrt 2$. Therefore, 
$\sup S = \max S = \sqrt{2}$ and
$\inf S = \min S = -1$. 

\item $S = \{x e^x: x < 0 \}$.

\sol Clearly, $xe^x<0$ for $x<0$. Therefore,
$\lim_{x \to 0} x e^x = \sup S = 0$ and
that does not belong to $S$. In order to find 
minimum of $xe^x$ take its derivative and set
to zero. We get $x=-1$. Obviously $x$ is allowed to be -1.
Therefore, $\inf S = \min S = -e^{-1}$.
\end{enumerate}

\item Let $T = \{ x: -x \in S \}$. Show that $- \sup T = \inf S$.

\sol Suppose $T$ has supermum. Then $\forall x \in T: x \le \sup T$.
Therefore, $\forall x \in T: -x \ge -\sup T$, i.e. $-\sup T$ is a lower bound
for $S$. We need to show $-\sup T$ is indeed the largest lower bound of $S$.

By contradiction, suppose $-\sup T$ is \textit{not} indeed the largest lower bound of $S$.
Therefore, there exist a ``$-c$'' such that $-\sup T < -c \le -x$.
Therefore, $\sup T > c \ge x$, i.e. $c$ is the upper bound for $T$
that is smaller than $\sup T$. This is a contradiction, and therefore, 
our assumption ($-\sup T$ is \textit{not} indeed the largest 
lower bound of $S$) is false.

\item Let $x, y \in \mathbb{R}$ where $x<y$.
Show that there is a rational number $z$ such that
$x < z < y$. 
Hint: By the archimedean property there is a positive integer $q$
such that $q^{-1} < y - x$. Let $z =  p/q$ where $p$ is smallest positive integer
such that $qx<p$.

\sol $y-x>0 \rightarrow \exists q \in \mathbb{N}~s.t.~ 0< 1/q < y-x$. 
Let $q$ be the smallest integer for which we have $1/q < y-x$.
Again by archimedean: $1/q>0, x>0, \exists p\in \mathbb{N} ~s.t.~ x < p/q$.
Let $p$ be the smallest integer for which we have $x<p/q$.
We only need to show $p/q<y$.

We know $1/q<y-x$ and $(p-1)/q < x$. Therefore, $p/q < y$.
\end{enumerate}

%%%%%%%%%%%%%%%%%%%%%%%%%%%%%%%%%%%%%%%%
%%%%%%%%%%%%%%%%%%%%%%%%%%%%%%%%%%%%%%%%
%%%%%%%%%%%%%%%%%%%%%%%%%%%%%%%%%%%%%%%%
\subsection{Euclidean $E^n$}
\textbf{Problems}
\begin{enumerate}
\item Let $n=4,~\mathbf{x}=(1, -1, 0, 2), ~\mathbf{y}=(3, -1, 1, 1)$.
Find $\mathbf{x+y, ~x-y, ~\norm{x+y},~ \norm{x-y}, ~\norm{x},~ \norm{y}, ~x.y}$. Verify (1.1)
and (1.2) for this example.

\sol \\
$\mathbf{x+y} = (4, -2, 1, 3)$, \\
$\mathbf{x-y} = (-2, 0, -1, 1)$, \\ 
$\mathbf{|x+y|} = \sqrt{16 + 4 + 1 + 9} = \sqrt{30}$,  \\
$\mathbf{|x-y|} = \sqrt{4 + 0 + 1 + 1} = \sqrt{6}$,  \\
$\mathbf{|x|} = \sqrt{1 + 1 + 0 + 4} = \sqrt{6}$,  \\
$\mathbf{|y|} = \sqrt{9 + 1 + 1 + 1} = \sqrt{12}$,  \\
$\mathbf{\langle x, y \rangle} = 1 \times 3 + (-1) \times (-1) + 0 \times1 +  2\times1 = 6$,  \\

(1.1) is Cauchy's inequality: $\norm{ \mathbf{\langle x, y \rangle}} \leq \norm{\mathbf{x}} . \norm{\mathbf{y}}$. 
In this example we see 
$\norm{ \mathbf{\langle x, y \rangle} } = 6 \leq  \sqrt{6} \times \sqrt{12} = 
\sqrt{6 \times 12} = \sqrt{6 \times 6 \times 2} = 6  \sqrt{2} = \norm{\mathbf{x}} . \norm{\mathbf{y}}$.

(1.2) is triangle inequality: 
$\norm{\mathbf{x} + \mathbf{y}} \leq \norm{\mathbf{x}} + \norm{\mathbf{y}}$.
In this problem we have $\norm{\mathbf{x}+\mathbf{y}} = \sqrt{30} = 5.47 \leq \sqrt{6} + \sqrt{12} = 7.92$.

\item Prove that standard euclidean inner product in $E^n ~(=\mathbb{R}^n)$
has the following four properties:
\begin{enumerate}[label=(\alph*)]
\item $\mathbf{\langle x, y \rangle} = \mathbf{\langle y, x \rangle}$
\item $\mathbf{\langle x+y, z \rangle} = \mathbf{\langle x, z \rangle} + \mathbf{\langle y, z \rangle}$
\item $\langle c\mathbf{x, y \rangle} = c\mathbf{\langle x, y \rangle}$
\item $\langle \mathbf{x, x \rangle} > 0$ if $\mathbf{x} \ne \mathbf{0}$
\end{enumerate}
\sol \\
\begin{enumerate}[label=(\alph*)]
\item $\mathbf{\langle x, y \rangle} = \sum_i x_i y_i = \sum_i y_i x_i = \mathbf{\langle y, x \rangle}$
\item 
\begin{gather*} % gather and aligned leads to having one label for eq.
\begin{aligned}
\mathbf{\langle x+y, z \rangle} &= \sum_i (x_i + y_i) z_i \\
                                                &= \sum_i x_i z_i + \sum_i y_i  z_i \\
                                                &= \mathbf{\langle x, z \rangle} + \mathbf{\langle y, z \rangle}
\end{aligned} % gather and aligned leads to having one label for eq.
\end{gather*}

\item $\langle c\mathbf{x, y \rangle} = \sum_i (c  x_i )  y_i = \sum_i c  x_i  y_i = c \sum_i x_i  y_i = 
c\mathbf{\langle x, y \rangle}$

\item $\langle \mathbf{x, x \rangle} = \sum_i x_i x_i = \sum_i x_i^2$ which obviously 
is greater than zero if at least one of $x_i$'s is nonzero.
\end{enumerate}

\item Using problem 2 show that:
\[ \begin{split} \langle  \mathbf{w} +  c\mathbf{x} ,  \mathbf{y} +  d\mathbf{z} \rangle =  \langle \mathbf{w}, \mathbf{y} \rangle + c\langle \mathbf{x}, \mathbf{y} \rangle  + \\ d\langle \mathbf{w}, \mathbf{z} \rangle  +  cd\langle \mathbf{x}, \mathbf{z} \rangle \end{split} \]

\sol
\begin{gather*} % gather and aligned leads to having one label for eq.
\begin{aligned}
\begin{split}
\langle \mathbf{w} + c\mathbf{x} ,  \mathbf{y} +  d\mathbf{z} \rangle &= 
\langle \mathbf{w} + c\mathbf{x} , \mathbf{y} \rangle + \langle \mathbf{w} +  c\mathbf{x} , d\mathbf{z} \rangle \\
&=  \langle \mathbf{w}, \mathbf{y} \rangle + \langle c\mathbf{x}, \mathbf{y} \rangle + \\ 
        & ~~~~~~~~~~ \langle \mathbf{w}, d\mathbf{z}\rangle + \langle c\mathbf{x}, d\mathbf{z}\rangle \\ 
&=  \langle \mathbf{w}, \mathbf{y} \rangle + c \langle \mathbf{x}, \mathbf{y} \rangle + \\ 
& ~~~~~~~~~~ d \langle \mathbf{w}, \mathbf{z}\rangle + cd\langle \mathbf{x}, \mathbf{z}\rangle 
\end{split}
\end{aligned} % gather and aligned leads to having one label for eq.
\end{gather*} 

\item Show that $\sum_{i=1}^n |x_i| \leq \sqrt{n} \norm{\mathbf{x}}$
for any $\mathbf{x} = \begin{bmatrix} x_1, & x_2,  \dots,& x_n\end{bmatrix}$.
Hint: First assume $x_i \geq 0$ and then use Cauchy's inequality: 
$ | \langle \mathbf{x, y} \rangle | \leq \norm{\mathbf{x}} . \norm{ \mathbf{y}}$ 
with $\mathbf{y} = \mathbf{1}= \begin{bmatrix} 1, & 1,  \dots, &1\end{bmatrix}$

\sol The hint is the solution! From now on we will not see any hints.
\begin{proof}
Since on the lefthand side of the inequality we have 
absolute values of entries of $\mathbf{x}$ and on the righthand side
we have squares of these entries in the norm of $\mathbf{x}$
we can assume all the entries are positive. 
Let $\mathbf{y} = \mathbf{1} $; the vector of ones.
Then expand the Cauchy's inequality to see the result:

\begin{gather*} % gather and aligned leads to having one label for eq.
\begin{aligned}
| \langle \mathbf{x, y} \rangle |
&= | \langle \mathbf{x, \mathbf{1}} \rangle | \\
&= \left | \sum_i x_i . 1 \right | \\
&= \sum_i x_i  \\
&\leq \norm{\mathbf{x}} . \norm{ \mathbf{y}} ~~~ \justif{\quad}{by Cauchy}\\
&= \norm{\mathbf{x}} . \sqrt{n}
\end{aligned} % gather and aligned leads to having one label for eq.
\end{gather*} 
\end{proof}

\item Show that $2\norm{\mathbf{x}}^2 + 2 \norm{\mathbf{y}}^2 = \norm{\mathbf{x+y}}^2 + \norm{\mathbf{x - y}}^2$. What does this say about parallelograms?

\sol
\begin{gather*} % gather and aligned leads to having one label for eq.
\begin{aligned}
\norm{\mathbf{x+y}}^2 + \norm{\mathbf{x - y}}^2 
&= \langle \mathbf{x + y, x + y} \rangle + \langle \mathbf{x - y, x - y} \rangle \\ 
&= \langle \mathbf{x, x} \rangle + \langle \mathbf{x, y} \rangle + 
     \langle \mathbf{y, x} \rangle + \langle \mathbf{y, y} \rangle + \\
  &~~~~  \langle \mathbf{x, x} \rangle - \langle \mathbf{x, y} \rangle - 
      \langle \mathbf{y, x} \rangle + \langle \mathbf{y, y} \rangle \\
&= 2\langle \mathbf{x, x} \rangle + 2\langle \mathbf{y, y} \rangle \\
&= 2 \norm{x}^2 + 2 \norm{y}^2
\end{aligned} % gather and aligned leads to having one label for eq.
\end{gather*} 

\item Show that 
$\norm{\mathbf{x+y}} . \norm{\mathbf{x - y}} \leq \norm{\mathbf{x}}^2 + \norm{\mathbf{y}}^2$. 
What does this say about parallelograms?

\sol
\begin{gather*} % gather and aligned leads to having one label for eq.
\begin{aligned}
\norm{\mathbf{x+y}} . \norm{\mathbf{x - y}} 
&= 
\end{aligned} % gather and aligned leads to having one label for eq.
\end{gather*} 


\end{enumerate}

%%%%%%%%%%%%%%%%%%%%%%%%%%%%%%%%%%%%%%%%
%%%%%%%%%%%%%%%%%%%%%%%%%%%%%%%%%%%%%%%%
%%%%%%%%%%%%%%%%%%%%%%%%%%%%%%%%%%%%%%%%
\subsection{Elementary geometry of $E^n$}



%%%%%%%%%%%%%%%%%%%%%%%%%%%%%%%%%%%%%%%%
%%%%%%%%%%%%%%%%%%%%%%%%%%%%%%%%%%%%%%%%
%%%%%%%%%%%%%%%%%%%%%%%%%%%%%%%%%%%%%%%%
\subsection{Basic topological notions in $E^n$}



%%%%%%%%%%%%%%%%%%%%%%%%%%%%%%%%%%%%%%%%
%%%%%%%%%%%%%%%%%%%%%%%%%%%%%%%%%%%%%%%%
%%%%%%%%%%%%%%%%%%%%%%%%%%%%%%%%%%%%%%%%
\subsection{Convex sets}