\chapter{Spaces: An Introduction to Analysis}
%%%%%%
%%%%%%  01_Exercises
%%%%%%
\section{Proofs, Sets, and Functions}
\label{Proofs-Sets-and-Functions}

\subsection{Families of sets}

\textbf{Problems}
\begin{enumerate}

\item If $X$ is a nonempty set and $\mathcal{A}$ is a family 
of subsets of $X$, we call $\mathcal{A}$ an algebra of 
sets if the following three properties are satisfied:

\begin{enumerate}[label=(\roman*)]
\item $\emptyset \in \mathcal{A}$.
\item If $A \in \mathcal{A}$ , then $A^c \in \mathcal{A}$ 
(all complements are with respect to the universe $X$;
hence $A^c = X \textbackslash A$). 

\item If $A, B \in \mathcal{A}$, then $A \cup B \in \mathcal{A}$.
\end{enumerate}
In the rest of the problem, we assume that $\mathcal{A}$ 
is an algebra of sets on $X$.

\begin{enumerate}
\item Show that $X \in \mathcal{A}$.
\item Show that if $A_1, A_2, \dots, A_n \in \mathcal{A}$ for an $ n \in \mathbb{N}$, then $\cup_{i=1}^n A_i \in  \mathcal{A}$.

\item Show that if $A_1, A_2, \dots, A_n \in \mathcal{A}$ for an $ n \in \mathbb{N}$, then $\cap_{i=1}^n A_i \in \mathcal{A}$.
\end{enumerate}

\sol
\begin{enumerate}
\item By definition--property (i)--we know $\emptyset \in \mathcal{A}$. By property (ii) we have: $X = \emptyset^c  \in \mathcal{A}$.
\item We know $A_1, A_2 \in \mathcal{A}$. Property (iii) 
$A_1 \cup A_2 \in \mathcal{A}$. 
Assume $B=\cup_{i=1}^{n-1} A_i \in \mathcal{A}$.
Then, by property (iii), again, $B \cup A_n  =  \cup_{i=1}^{n} A_i  \in \mathcal{A}$. 


\item Let $A_1, A_2, \dots, A_n \in \mathcal{A}$. By property (ii) we have
$A_1^c, A_2^c, \dots, A_n^c \in \mathcal{A}$.
By property (iii) we have $\cup_{i=1}^{n} A_i^c  \in \mathcal{A}$.
Therefore, by property (ii) we know $(\cup_{i=1}^{n} A_i^c)^c  \in \mathcal{A}$.
By De Morgan's law 
$ (\cup_{i=1}^{n} A_i^c)^c = \cap_{i=1}^{n} A_i  \in \mathcal{A}$.
\end{enumerate}



\end{enumerate}
