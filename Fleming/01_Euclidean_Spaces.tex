%%%%%%
%%%%%%  01_Exercises
%%%%%%
\section{Euclidean Spaces}

\subsection{The real number system}

\textbf{Problems}
\begin{enumerate}
\item Find $\sup S$ and $\inf S$ for the following sets
and state whether $\sup S$ and $\inf S$ are elements of $S$.
\begin{enumerate}[label=(\alph*)]
\item $S = \{x: x^2-3x+2 < 0 \}$.

\sol The roots of $x^2-3x+2 = 0$ are 1 and 2.
The function is concave up. So, $\inf S = 1$ and $\sup S = 2$
and obviously they do not belong to $S$.

\item $S = \{x: x^3 +x^2 - 2x \le 2 \}$.

\sol Let $f(x) = x^3 +x^2 - 2x - 2$. 
We are looking for values $x$ where $f(x) \le 0$.
We can rewrite $f(x) = (x^2 - 2) (x + 1)$. Roots of
$f(x)$ are $-1, \pm\sqrt{2}$.
From the figure of the function $\sup S = \max S = \sqrt{2}$.
And similarly, or by a close look at the property given
in definition of $S$ we can guess $\inf S = - \infty$.

\item $S = \{ sin x + cos x : x \in [0, \pi] \}$.

\sol Obviously for $x \in [0, \pi]$ we have
$ -1 \le sin x +cos x \le \sqrt 2$. Therefore, 
$\sup S = \max S = \sqrt{2}$ and
$\inf S = \min S = -1$. 

\item $S = \{x e^x: x < 0 \}$.

\sol Clearly, $xe^x<0$ for $x<0$. Therefore,
$\lim_{x \to 0} x e^x = \sup S = 0$ and
that does not belong to $S$. In order to find 
minimum of $xe^x$ take its derivative and set
to zero. We get $x=-1$. Obviously $x$ is allowed to be -1.
Therefore, $\inf S = \min S = -e^{-1}$.
\end{enumerate}

\item Let $T = \{ x: -x \in S \}$. Show that $- \sup T = \inf S$.

\sol Suppose $T$ has supermum. Then $\forall x \in T: x \le \sup T$.
Therefore, $\forall x \in T: -x \ge -\sup T$, i.e. $-\sup T$ is a lower bound
for $S$. We need to show $-\sup T$ is indeed the largest lower bound of $S$.

By contradiction, suppose $-\sup T$ is \textit{not} indeed the largest lower bound of $S$.
Therefore, there exist a ``$-c$'' such that $-\sup T < -c \le -x$.
Therefore, $\sup T > c \ge x$, i.e. $c$ is the upper bound for $T$
that is smaller than $\sup T$. This is a contradiction, and therefore, 
our assumption ($-\sup T$ is \textit{not} indeed the largest 
lower bound of $S$) is false.

\item Let $x, y \in \mathbb{R}$ where $x<y$.
Show that there is a rational number $z$ such that
$x < z < y$. 
Hint: By the archimedean property there is a positive integer $q$
such that $q^{-1} < y - x$. Let $z =  p/q$ where $p$ is smallest positive integer
such that $qx<p$.

\sol 
\end{enumerate}


\subsection{Euclidean $E^n$}
\subsection{Elementary geometry of $E^n$}
\subsection{Basic topological notions in $E^n$}
\subsection{Convex sets}