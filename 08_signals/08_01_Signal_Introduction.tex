\chapter{Signal and Image Processing}
\label{chap:Signal_and_Image_Processing}

\section{Introduction}
\label{sec:Signal_and_Image_Processing_Introduction}
How~\citep{SingleBaraniuk} do we extract usable information from 
data, especially data that is full of noise and or information 
that has nothing to do with the information we are interested in extracting?

That this is an old challenge is underlined by the fact that the first
appearance of the phrase ``Looking for a needle in a haystack'' is
nearly 500 years old.\marginnote{First known appearance of the idiom
``Looking for a needle in a haystack'', in written English, is from
St.\@ Thomas More, in 1532. The actual quote is ``to go looking for a
needle in a meadow''.}  But the fact that life introduces
confounding data -- entropy, noise, etc -- has made this idiom a part
of the working wisdom of anyone trying to make careful inferences,
implying the idea is much, much older.

\section{Definitions}
\label{sec:definition}
%
% definition of sparsity
%

\begin{deff}{}
\textbf{Sparsity:} T.
\end{deff}{}

%
% definition of Overcomplete Dictionary
%

\begin{deff}{}
\textbf{Overcomplete Dictionary:} T.
\end{deff}{}

%
% relation between sparsity and overcomplete dictionary
%

\textbf{Relation Between Sparsity and Overcompleteness:} T.



% \addcontentsline{toc}{section}{Definitions}
\begin{tcolorbox}
  \textbf{Matching Pursuit} is a simple but effective algorithm that looks for approximate solutions to:
  %
  \begin{gather*}
    \min_{x\in\R^m} \norm{\vect{x}}_0
    \qquad\text{given}\qquad
    \norm{\vect{s} - \mat{D}\cdot\vect{x}}_2
    \leq \epsilon\norm{\vect{r}}_2.
  \end{gather*}
  %
  Let $\vect{s}_j$ be a sequence of approximations approaching the signal $\vect{s}$ with residuals $\vect{r}_j = \vect{s} - \vect{s}_j$:
  %
  \begin{gather*}
    \vect{s}_{j} = \sum_{i=1}^{j} \alpha_i\vect{d}_i.
  \end{gather*}
  %
  The \textbf{Matching Pursuit} algorithm can be expressed as:
  %
  \begin{enumerate}
    \item Prepare your dictionary as a map from residuals $\vect{r}$ to normalized vectors $\vect{d}$ which are the best match in the sense of maximizing the overlap $\abs{\braket{\vect{d}, \vect{r}}}$:
     %
     \begin{gather*}
       \vect{d} = \match(\vect{r})
                = \argmax_{\norm{\vect{d}}_{2} = 1}
                  \abs{\braket{\vect{d}, \vect{r}}}.
     \end{gather*}
   \item Choose a stopping criterion $\epsilon < 1$.  E.g.\@ $\epsilon = 0.05 = 5\%$.
   \item Start with residual $\vect{r}_1 = \vect{s}$ and iterate until $\norm{\vect{r}_j}_2 < \epsilon\norm{\vect{s}}_2$, or some maximum rank is reached:
   %
   \begin{align*}
     \vect{d}_j &= \match(\vect{r}_j), &
     \alpha_j &= \braket{\vect{d}_j,\vect{r}_j}, &
     \vect{r}_{j+1} &= \vect{r}_j - \vect{d}_{j}\alpha_j.
   \end{align*}
 \end{enumerate}
\end{tcolorbox}


\begin{tcolorbox}
{\tblue \textbf{The Scientific Goal} -- find $h_C$, given $e_M$ (and knowledge of $P$):
\[h_C = P^{-1}(e_M)\]
where $P^{-1}$ is the inverse of $P$.}
\end{tcolorbox}

\begin{tcolorbox}
\begin{quotation}
\noindent\textbf{Modern Occam:} \textit{The right basis for a signal is the basis in which the signal representation is sparse}.
\end{quotation}
\end{tcolorbox}

