\chapter{Metric Design}
\label{chap:Metric Design}

In business there are metrics to evaluate performance of the model.
Of course, since it is business, they care about money.
Thus, the metric they want to hear is how much revenue is increased.
These metrics may not be convex, smooth, etc. So, they may not be
used during training phase. 
But, hopefully, the metrics used during the training phase
are good surrogate for the business-metric. 
Lets see what is out there.
At this point, Nov. 2, 2025, these are from edicative.io course; \emph{Machine Learning System Design}.
Hopefully, I will remember to cite everything properly.

Metrics used during training are referred to by \emph{offline metrics.}

\section{Educative.io Metrics}
\subsection{AirBnB}


{\bf Offline Metrics}

\begin{description}
\item [Normalized discounted Cumulative Gain: nDCG] is a standard metric in ranking problems where position matters. It gives higher weight to correct predictions near the top of the list—exactly what we want in search ranking.


\begin{tcolorbox}[colback=aliceblue!10!white,colframe=blue!70!black,title=\textbf{Why nDCG?}]
% olback=green!10!white,colframe=green!80!black
Users rarely scroll through all results. A relevant result at position 2 is more valuable than at position 10. 
It accounts for both relevance and position, unlike basic accuracy or AUC. 
It reflects user satisfaction better than simple classification metrics like precision or recall.
\end{tcolorbox}

\begin{align}
\text{DCG}_p &= \sum_{i=1}^{p} \frac{\text{rel}_i}{\log_2(i+1)} \\[6pt]
\text{nDCG}_p &= \frac{\text{DCG}_p}{\text{IDCG}_p},
\end{align}
$\text{rel}_i$= relevance score of the result at position $i$ and 
IDCG = ideal DCG, if all relevant results were perfectly ordered\sidenote{We need to expand on these. What are they?}



\end{description}