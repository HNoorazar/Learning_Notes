%%%%%%
%%%%%%  00_Motivation
%%%%%%
\section*{Background \& Motivation}
I am starting to learn and immerse myself in
machine learning and image processing field.
I decided to take notes and collect everything new and
useful I learn. Perhaps, I can combine them in one
single file and some people find it useful.
\textbf{\textit{These notes are largely verbatim from the sources.}}
\marginnote{The reason is that there are things that I like very much
but have not had a chance to learn and work on them. 
Hence, I will collect whatever I learn in these notes.}

There will be at least 4 parts to this series; regression,
classification, clustering, deep learning, and last but 
not the least, signal/image processing. Another file
will be devoted for the field of analysis.
\iffalse
Personally, I do not want to do deep learning until
I have a chance to work in a team with experts in the field. 
The reason is it seems there is no logic to the structure of 
networks. I prefer to learn by getting my hands dirty. However, 
I do like the convolutional neural networks with application in 
images and videos and will play with them if I have time.
\fi

On a different note, I have not seen more than a couple of
math departments. So, I have a limited experience
how math departments work other than the WSU.
I think there is no ``program'' in most math the departments.
Individuals teach whatever they feel and students are like headless chickens
wandering around till they land on a field. If a variety of 
possibilities are not exposed to them, their choices might not be optimal.
I think there can be
distinct and goal oriented programs. I personally like the
following (\cref{tab:CoursePlan}) and will try to learn the topics on my own.
This will built a solid foundation either for working in 
these areas or for moving forward to some other disciplines
such as machine learning--both from statistical point of view
or otherwise-- image analysis, etc.
During the first month of graduate school or first summer
there should be a compact
collaborative work around
practical uses of science (or mathematics in this case) 
in real life. Or perhaps the first semester can be spent
with different groups.
For example, I enjoyed very much doing 
\href{https://static1.squarespace.com/static/5a4c161cfe54ef45b17aa18e/t/6058567c9ce4f72ce5955609/1616402050225/cycles-discovery-metrics-forbes-noorazar-sandine-vixie.pdf}{this explorations}
for the \href{https://journal.cycles.org/Issues/Vol50-No4-2021/}{Cycles.org} 
magazine. Another benefit is that one tastes different
flavors in a short period of time. Such experiences
could preferably start during undergraduate period. 

Put differently, most of the programs and schools
are designed in a bottom-up fashion. Let me clarify.
Students may learn what an integral is
or what is the probability of a flipped coin landing on
its head or tail. But it may take a long long time to 
realize integral is an integral part of everything we do.
Integrals are in measure theory, which is foundation of probability 
theory, which is one of the important bases for data science.
Therefore, having a short collaborative workshops
that goes through applications\marginnote{Of course 
abstract math is very interesting and solving a problem for 
sake of the problem is very appealing
and satisfying. But I am not too sure Stanford has room 
to hire everybody.} 
of mathematics,
provide a top-down opportunity to have a
overview of the landscape from apex.
Spending so many years in the weeds is tiring.

Having a top-down approach, 
and experiencing 5 different direction in a short period,
seeing how the applicable technology depends
on theoretical science is useful to find
\begin{itemize}
\item the sweet spot one wants to be in; 
on the spectrum ranging from technology to abstract fields, 
a more comfortable life to a less comfortable life.

\item which direction to go at a given sweet spot; e.g. on pure side 
there are linear algebra, topology, etc.
(I mentioned ``application'' a few times, but the abstract side is 
must be an inseparable part of the experience. \href{https://static1.squarespace.com/static/5a4c161cfe54ef45b17aa18e/t/6058567c9ce4f72ce5955609/1616402050225/cycles-discovery-metrics-forbes-noorazar-sandine-vixie.pdf}{Here is the example, again.})
\end{itemize} 

Most probably, I think, the environment you are at,
does not provide such opportunity. Therefore, you have to seek
it. Either find summer schools to attend to (and pay the price of moving 
and living in a new place for a short period), or 
at least find papers of this nature to read.


On the other hand, one need
to see clearly and determine the passion and need
to come up with a long-term plan; if we assume the taste
and direction will not change. Whether the interests change 
or not, the best approach 
is to put all the effort one can, in the current ongoing path.
Looking back and thinking ``I could have done better,'' is
very painful and consequences are immeasurable.


\begin{table}
\caption{Course Plan} 
\label{tab:CoursePlan}
\begin{tabular}{llr}
\cline{1-3}
\hline
\rowcolor{shadecolor} & \textbf{Course} & \textbf{Min. Req.}\\
\hline
\textbf{1st Year}    & Real Analysis   & 1 \& 2 \\
         &\cellcolor{aliceblue}Mesure \& Integration & \cellcolor{aliceblue} 1 \& 2 \\
         
\textbf{2nd Year}    & Functional Analysis   & 1 \& 2 \\
         &\cellcolor{aliceblue}Linear Algebra & \cellcolor{aliceblue} 1 \& 2 \\

\textbf{3rd Year}    & Topology and Manifolds   & 1 \& 2 \\
         &\cellcolor{aliceblue}Wavelets and Fourier & \cellcolor{aliceblue} 1 \& 2 \\
\hline
\end{tabular}
\end{table}

\vspace{.1in}
Beyond the technical part of school I would advise (students)
to have a creative hobby in which you create something;
a piece of music, a pottery item, or wood work projects.
Creating science is not much different from creating these objects.
The time is taken to let the imagination and reasoning
walk through unseen places. 
Spending time in the weeds and microscopic levels of 
courses, the way mathematical ideas are taught or written,
or life in general (e.g. being stuck in an unpleasant situation), 
can kill the creativity and soul.
Fear of failing or being exposed may run deep even without
realizing when or how it happened. 
Doing handwork projects can nourish creativity and
be a peaceful outlet, keeping the child-like spirit alive. 

\section*{Structure of My Notes}
I will try to write the notes so that it is approachable
to everyone; with or without mathematics backgrounds.
The not-too-technical parts will be written in boxes. 
(Well, I forgot about this part! Perhaps I need to
build a solid understanding for myself first!)

