%%%%%%
%%%%%%  00_Motivation
%%%%%%
\section*{Motivation/Background}
I am starting to learn and immerse myself in
machine learning and image processing field.
I decided to take notes and collect everything new and
useful I learn. Perhaps, I can combine them in one
single file and put it on arXiv eventually.
\textbf{\textit{These notes are verbatim from the sources.}}

The reason is that there are things that I like very much
but have not had a chance to learn and work on them. 
Hence, I will collect whatever I learn in these notes. 

There will be at least 4 parts to this series; regression,
classification, clustering, deep learning, and last but 
not the least, signal/image processing.

Personally, I do not want to do deep learning until
I have a chance to work in a team with experts in the field. 
The reason is it seems there is no logic to the structure of 
networks. I prefer to learn by getting my hands dirty. However, 
I do like the convolutional neural networks with application in 
images and videos and will play with them if I have time.

On a different note, I have not seen different
math departments. So, I have a limited experience
how math departments work other than the WSU.
I think there is no ``program'' in the department.
Individuals teach whatever they feel. I think there can be
distinct and goal oriented programs. I personally like the
following (\cref{tab:CoursePlan}) and will try to learn the topics on my own.
This will built a solid foundation either for working in 
these areas or for moving forward to some other disciplines
such as machine learning--both from statistical point of view
or otherwise-- image analysis, etc.
During the first month of grad. school or first summer
there should be a compact collaborative work around
practical uses in real life and future. For example,
I enjoyed very much doing \href{https://static1.squarespace.com/static/5a4c161cfe54ef45b17aa18e/t/6058567c9ce4f72ce5955609/1616402050225/cycles-discovery-metrics-forbes-noorazar-sandine-vixie.pdf}{this explorations}
for the \href{https://journal.cycles.org/Issues/Vol50-No4-2021/}{Cycles.org} 
magazine. Another benefit is that one tastes different
flavors in a short period of time. But again, some of this perhaps must start during 
undergraduate period. On the other hand, one need
to see and determine clearly the passion and need
to come up with a long-term plan; if we assume the taste
and direction will not change.
\begin{table}
\caption{Course Plan} 
\label{tab:CoursePlan}
\begin{tabular}{llr}
\cline{1-3}
\hline
\rowcolor{shadecolor} & \textbf{Course} & \textbf{Min. Req.}\\
\hline
\textbf{1st Year}    & Real Analysis   & 1 \& 2 \\
         &\cellcolor{aliceblue}Mesure \& Integration & \cellcolor{aliceblue} 1 \& 2 \\
         
\textbf{2nd Year}    & Functional Analysis   & 1 \& 2 \\
         &\cellcolor{aliceblue}Linear Algebra & \cellcolor{aliceblue} 1 \& 2 \\

\textbf{3rd Year}    & Topology and Manifolds   & 1 \& 2 \\
         &\cellcolor{aliceblue}Wavelets and Fourier & \cellcolor{aliceblue} 1 \& 2 \\
\hline
\end{tabular}
\end{table}

\section*{Structure of My Notes}
I will try to write the notes so that it is approachable
to everyone; with or without mathematics backgrounds.
The not-too-technical parts will be written in boxes.