\section{Tech Papers}
\label{sec:Tech_Papers}
The first paper from Tech I am reading is Baysian CTRprediction~\cite{graepel2010web}.
At least the first one I am taking notes on in this Notes. 
Interesting and new things in it and its method replaced Bing's CTR prediction algorithm in 2009.

\begin{deff}
We refer to an ad shown to a particular user in a particular
page view as an ad impression
\end{deff}

\begin{description}

\item [Contributions]
First, it
describes the Sponsored Search application scenario, the
key role of CTR prediction in general, and the particular
constraints derived from the task, including accuracy,
calibration, scalability, dynamics, and exploration.
Second, it describes a new Bayesian online learning
algorithm for binary prediction, subsequently referred to
as adPredictor. The algorithm is based on a generalised
linear model with a probit (cumulative Gaussian) link
function, a factorising Gaussian belief distribution on the
feature weights, and calculates the approximate posterior
using message passing, providing simple, closed-form
update equations with automatic feature-wise learning
rate adaptation. Third, we discuss the techniques we
employed to make adPredictor work in Bing’s production
environment, now driving 100% Sponsored Search traffic
with ad impressions per year.

\item [Keyword Auction]  is done as follows
\begin{enumerate}
\item For a given product advertisers identify suitable keywords
likely to be typed by users. 

\item For each of those keywords the advertisers provide a bid. 

\item When a user types a query, the search
engine matches the keywords of all the advertisers against
the query.

\item The search engine needs to allocate the available ad
positions to the ads in the auction and needs to determine
appropriate payments. This is achieved by a mechanism
referred to as a Generalized Second Price (GSP) Auction.
\end{enumerate}

\item [Features] can be grouped in 3 categories:
\begin{enumerate}
\item {\bf Ad. features}
bid phrases, ad title, ad text, landing page
URL, landing page itself, and a hierarchy of advertiser,
account, campaign, ad group and ad. 

\item {\bf Query features} 
include the search keywords, possible algorithmic query
expansion, cleaning and stemming.

\item {\bf Context features}  
include display location, geographic location, time, user
data and search history.
\end{enumerate}

\item [Scalability] of the algorithm is
ensured through a principled weight pruning
procedure and an approximate parallel
implementation.

\item [Advertisement Players Objectives] are three in this paper.
advertisers, users, and the
search engine. These three types of players have different,
even contradictory objectives. Advertisers are interested
in maximising their return on investment at high volume.
Users would like to see maximally relevant ads that help
them pursue their intent. The search engine would like to
maximize revenue and growth.

Internally, these conflicting goals are mapped to different
key performance indicators (KPIs) that are used to tune
the ad selection system. However, these KPIs are
influenced by a large number of other subsystems such as
fraud detection, query expansion, keyword-query
matching, etc. Furthermore, there are a large number of
parameters influencing the KPIs including reserve prices
and rank-score parameters. So, while the ultimate test of a
CTR predictor lies in its performance as part of the ad
selection system, in a modular architecture it is often best
to identify isolated performance measures as proxies for
in-system performance.

\item [Distribution changes over time] due to seasonality, 
change of taste, changes in web content, economic conditions, etc. 
Online algorithms, batch learning on shifting windows.


\item [Exploration vs. Exploitation] Interesting point: The ads shown
to users is decided by the algorithm that predicts which ads are good.
But then, those will be our training set for future. So, the trade-off between
exploration and exploitation should be considered.
\end{description}