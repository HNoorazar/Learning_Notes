Structured Query Language. Relational database.\marginnote{SQL from Coursera: UC Davis. SQLite here.}
Read. Write. Update. (no copy. no creation.)

\begin{enumerate}
\item List column types of a given table in SQLite: PRAGMA should be capital letters. 
\begin{lstlisting}
PRAGMA table_info(table_name); 
\end{lstlisting}

\item Be careful with numbers. Divide by 100 vs 100.00.
\end{enumerate}

\begin{table}[h!]
\centering
\begin{tabular}{c l p{8cm}}
\hline
\textbf{Step} & \textbf{Clause} & \textbf{Purpose} \\
\hline
1 & \texttt{FROM} & Identify and combine tables (including joins and subqueries). \\
2 & \texttt{WHERE} & Filter rows \emph{before} grouping. \\
3 & \texttt{GROUP BY} & Group remaining rows by specified columns. \\
4 & \texttt{HAVING} & Filter groups \emph{after} grouping. \\
5 & \texttt{SELECT} & Compute expressions and aggregates (such as \texttt{AVG}) for each group. \\
6 & \texttt{ORDER BY} & Sort the final results. \\
\hline
\end{tabular}
\caption{Logical order of SQL query processing}
\end{table}


Some Queries are on \href{https://docs.google.com/document/d/10g0QW-8TP935AQhp6Wcjyli3I-zSqmvAGqt1ORVYnio/edit?usp=sharing}{this google doc}.


\begin{exm} Retained customers
\begin{lstlisting}[language=SQL]
SELECT 
    activity.date,
    COUNT(DISTINCT activity.user_id) AS active_users,
    COUNT(DISTINCT future_activity.user_id) AS retained_users,
    COUNT(DISTINCT future_activity.user_id)::FLOAT /   
                     COUNT(DISTINCT activity.user_id) AS retention
FROM activity LEFT JOIN activity AS future_activity
    ON activity.user_id = future_activity.user_id
   AND activity.date = future_activity.date - INTERVAL '1 day'
GROUP BY activity.date
ORDER BY activity.date;
\end{lstlisting}
\end{exm}


\begin{exm} Count and show result in one row:
\begin{lstlisting}[language=SQL]
SELECT
    COUNT(CASE WHEN gender = 'M' THEN 1 END) AS male_count,
    COUNT(CASE WHEN gender = 'F' THEN 1 END) AS female_count
FROM patients;
\end{lstlisting}
\end{exm}







\begin{exm} (Medium) 
Show number of patients per city sorted by patient count and then city.
\begin{lstlisting}[language=SQL]
SELECT city, COUNT(DISTINCT patient_id) as num_patients
FROM patients
GROUP BY city
ORDER BY num_patients DESC, city ASC
\end{lstlisting}
\end{exm}

\begin{exm}  (Medium) 
Show first name, last name and role of every person that is either patient or doctor.
The roles are either ``Patient'' or ``Doctor''.
\begin{figure}[ht]
\centering
\subfloat[]{\includegraphics[width=0.25\textwidth]{00_figures/SQL/patients}}\hspace{0.5cm}
\subfloat[]{\includegraphics[width=0.3\textwidth]{00_figures/SQL/doctors}}\hspace{0.5cm}
\subfloat[]{\includegraphics[width=0.25\textwidth]{00_figures/SQL/EO1}}
\caption[11\baselineskip]{Patients, doctors, and expected output.\label{fig:SQLEg1}}
\end{figure}

\begin{lstlisting}[language=SQL]
SELECT first_name, last_name, 'Patient' as role 
FROM patients union all (SELECT first_name, last_name, 'Doctor' 
FROM doctors);
\end{lstlisting}
\end{exm}



\begin{exm} 
(Medium) Show all allergies ordered by popularity. 
Remove NULL values from query. See \cref{fig:SQLEg1} for \emph{patients} table.
\begin{lstlisting}[language=SQL]
SELECT allergies, COUNT(DISTINCT patient_id) AS total_diagnosis
FROM patients
GROUP BY allergies
Having allergies NOT null
ORDER BY total_diagnosis DESC
\end{lstlisting}
\end{exm}

\begin{exm} (Medium)
Show all patient's first name, last name, and birth date who were born in the 1970s decade. 
Sort the list starting from the earliest birth date.
\begin{lstlisting}[language=SQL]
SELECT first_name, last_name, birth_date
FROM patients
WHERE YEAR(birth_date) BETWEEN 1970 AND 1979
-- WHERE FLOOR(YEAR(birth_date) / 10) = 197
RDER BY birth_date ASC;
\end{lstlisting}
\end{exm}


\begin{exm} T following 2 make the same result:
\begin{lstlisting}[language=SQL]
SELECT patient_id, weight, height, 
  CASE  WHEN (weight / power(height/100.0, 2)) >= 30 then 1 else 0  END as isObese
FROM patients

SELECT patient_id,  weight,  height,
  weight / power(CAST(height AS float) / 100, 2) >= 30 AS obese
FROM patients

# Shortest: combined two above:
SELECT patient_id,  weight,  height,
  weight / power(height/ 100.00, 2) >= 30 AS obese
FROM patients
\end{lstlisting}
\end{exm}


\begin{exm} Difficult
\begin{lstlisting}[language=SQL]
SELECT  p.patient_id,  p.first_name,  p.last_name, 
                    ph.specialty AS attending_doctor_specialty
FROM patients p  
            JOIN admissions a ON a.patient_id = p.patient_id
            JOIN doctors ph ON ph.doctor_id = a.attending_doctor_id
WHERE ph.first_name = 'Lisa' and a.diagnosis = 'Epilepsy'
--------------
SELECT p.patient_id,  p.first_name,  p.last_name,  doc1.specialty
FROM patients p JOIN (
       SELECT *  
       FROM admissions JOIN doctors 
      ON admissions.attending_doctor_id = doctors.doctor_id ) AS
doc1 USING (patient_id)
WHERE  doc1.diagnosis = 'Epilepsy' AND doc1.first_name = 'Lisa'
--------------
SELECT p.patient_id,  p.first_name,  p.last_name,  doctors.specialty
FROM patients p,  doctors,  admissions
WHERE  p.patient_id = admissions.patient_id
     AND admissions.attending_doctor_id = doctors.doctor_id
     AND admissions.diagnosis = 'Epilepsy'
     AND doctors.first_name = 'Lisa';
\end{lstlisting}
\end{exm}


\begin{exm} Hi
\begin{lstlisting}[language=SQL]
\end{lstlisting}
\end{exm}




