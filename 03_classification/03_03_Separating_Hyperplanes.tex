\section{Separating Hyperplanes}
\label{sec:Separating_Hyperplanes}
Separating hyperplanes procedures construct linear decision boundaries 
that explicitly try to separate the data into different classes as well as possible.

%%%%%%%%%%%%%%%%%%%%%%%%%%%%%%%%
%%%%%%%%%%%%%%%%%%%%%%%%%%%%%%%%
%%%%%%%%%%%%%%%%%%%%%%%%%%%%%%%%

\subsection{Rosenblatt's Perceptron Learning Algorithm}
\label{sec:RosenblattPerceptron}
The perceptron learning algorithm tries to find a 
separating hyperplane by minimizing the distance of 
misclassified points to the decision boundary. If
a response $y_i = 1$ is misclassified, then 
$x_i^T \beta + \beta_0 < 0 $, and the opposite for
a misclassified response with $y_i = -1$. The goal is to minimize
\begin{equation}\label{eq:RosenblattsGoal}
D(\beta, \beta_0) = - \sum_{i \in \mathcal{M}} y_i (x_i^T\beta + \beta_0 )
\end{equation}
where $\mathcal{M}$ indexes the set of misclassified points. 
The quantity is non-negative and proportional to the distance 
of the misclassified points to the decision boundary defined 
by $\beta^T x_i+ \beta_0 = 0$. The gradient (assuming $\mathcal{M}$ is fixed) 
is given by
\begin{gather}
% gather and aligned leads to having one label for eq.
\begin{aligned}
\partial \frac{D(\beta, \beta_0)}{\beta} &= - \sum_{i \in \mathcal{M}} y_ix_i, \\
\partial \frac{D(\beta, \beta_0)}{\beta_0} &= - \sum_{i \in \mathcal{M}} y_i.\\
\end{aligned}
\end{gather}
The algorithm in fact uses stochastic gradient descent 
to minimize this piecewise linear criterion.
%%%%%%%%%%%%%%%%%%%%%%%%%%%%%%%%
%%%%%%%%%%%%%%%%%%%%%%%%%%%%%%%%
%%%%%%%%%%%%%%%%%%%%%%%%%%%%%%%%

\subsection{Optimal Separating Hyperplanes}
\label{sec:Optimal_Separating_Hyperplanes}
Optimal Separating Hyperplanes are the method that
will develop into the support vector machines.

The optimal separating hyperplane separates the 
two classes and maximizes the distance to the closest point from 
either class\cite{vapnik2013nature}.
Not only does this provide a unique solution to the separating 
hyperplane problem, but by maximizing the margin between 
the two classes on the training data, this leads to better 
classification performance on test data.

Consider the optimization problem
\begin{equation}
\label{eq:SepHyperOpt}
\begin{aligned}
& \max_{\beta, \beta_0, \norm{\beta}=1} M,\\
\text{subject to } & y_i(x_i^T \beta + \beta_0) \ge M,~i=1, \dots, N.\\
\end{aligned}
\end{equation}

The set of conditions ensure that all the points are at 
least a signed distance $M$ from the decision boundary 
defined by $\beta$ and $\beta_0$, and we seek the largest such $M$
and associated parameters. We can get rid of the 
$\norm{\beta} = 1$ constraint by replacing the conditions with
\begin{equation}
\begin{aligned}
\frac{1}{\norm{\beta}} y_i (x_i^T \beta + \beta_0) \ge M
\end{aligned}
\end{equation}
(which redefines $\beta_0$) or equivalently
\begin{equation}
\begin{aligned}
y_i (x_i^T \beta + \beta_0) \ge M \norm{\beta}
\end{aligned}
\end{equation}
Since for any $\beta$ and $ \beta_0$ satisfying these inequalities, 
any positively scaled multiple satisfies them too, we can 
arbitrarily set $ \beta = 1 / M $. Thus~\ref{eq:SepHyperOpt} 
is equivalent to
\begin{equation}\label{eq:SepHyperOpt2}
\begin{aligned}
& \min_{\beta, \beta_0} \frac{1}{2}\norm{\beta}^2,\\
\text{subject to } & y_i(x_i^T \beta + \beta_0) \ge 1,~i=1, \dots, N.\\
\end{aligned}
\end{equation}
The constraints define an empty slab or margin around 
the linear decision boundary of thickness $1/\norm{\beta}$. 
Hence we choose $\beta$ and $\beta_0$ to maximize its thickness. 
This is a convex optimization problem (\hl{quadratic criterion with linear inequality constraints}).
The Lagrange (primal) function, to be minimized w.r.t. $\beta$ and $\beta_0$, is
\begin{equation}\label{eq:LagrangeSepHyp}
\begin{aligned}
L_p = \frac{1}{2} \norm{\beta}^2 - \sum_{i=1}^N \alpha_i \left[ y_i(x_i^T \beta + \beta_0) - 1  \right]
\end{aligned}
\end{equation}
Setting the derivatives to zero, we obtain:
\begin{equation}\label{eq:someCond}
\begin{aligned}
\beta &= \sum_{i=1}^N\alpha_i y_i x_i,\\
    0   &= \sum_{i=1}^N\alpha_i y_i  
\end{aligned}
\end{equation}
and substituting these in~\ref{eq:LagrangeSepHyp} we obtain the so-called Wolfe dual
we obtain the so-called Wolfe dual
\begin{equation}\label{eq:WolfeSepHyp}
\begin{aligned}
L_D = \sum_{i=1}^N\alpha_i - \frac{1}{2} \sum_{i=1}^N \sum_{k=1}^N \alpha_i \alpha_k y_i y_k x_i^T x_k \\
\text{subject to } \alpha_i \ge 0, \text{ and }  \sum_{i=1}^N\alpha_i y_i  = 1
\end{aligned}
\end{equation}
The solution is obtained by maximizing $L_D$ in 
the positive orthant, a simpler convex optimization problem, 
for which standard software can be used.
In addition the solution must satisfy the Karush–Kuhn–Tucker conditions, 
which include \ref{eq:someCond},~\ref{eq:WolfeSepHyp} and
\begin{equation}
\begin{aligned}
\alpha_i \left[ y_i(x_i^T \beta + \beta_0) - 1  \right] = 0 ~ \forall i.
\end{aligned}
\end{equation}




