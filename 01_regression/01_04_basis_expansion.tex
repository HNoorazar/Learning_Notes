\section{Basis Expansion}
\label{sec:Basis_Expansion}

\begin{deff}{}
The additive models assume $f(X)=\sum_{j=1}^p f_j(x_j)$.
\end{deff}{}


\begin{deff}{}
Let $h_m(X)$ be a transformation of input variables $X$.
Then a linear basis expansion in $X$ is modeled as 
\begin{equation}\label{eq:linearbasisexpansion}
f(X) = \sum_{m=1}^M \beta_m h_m(X).
\end{equation}
\end{deff}{}
\noindent Once the basis functions $h_m$ have been determined, 
the models are linear in these new variables, and the 
fitting proceeds as before.

\subsection{Piecewise Polynomials and Splines}
\label{sec:PiecewisePolynomialsandSplines}
In this section and the sections~\cref{FilteringandFeatureExtraction,SmoothingSplines,AutomaticSelectionoftheSmoothingParameters,NonparametricLogisticRegression,MultidimensionalSplines} 
we assume $X$ is one-dimensional.

Figure~\ref{fig:pieceWiseCubic} shows a series of 
piecewise-linear polynomials.
\begin{figure*}[ht!]
%  \centering
  \captionsetup{justification=centering}
  \includegraphics[width=0.6\textwidth]{00_figures/pieceWiseLinear}
  \caption[][-18\baselineskip]{The top left panel shows a piecewise constant 
  function fit to some artificial data.
The broken vertical lines indicate the positions of the two knots 
$\xi_1$ and $\xi_2$. 
The blue curve represents the 
true function, from which the data were generated 
with Gaussian noise. 
The remaining two panels show piecewise linear 
functions fit to the same data—the top right 
unrestricted, and the lower left restricted 
to be continuous at the knots. The lower right panel shows a 
piecewise–linear basis function, $h_x(X) = (X - \xi_1)_+$,
continuous at $\xi_1$. The black points indicate the 
sample evaluations $h_3(x_i)$, $i = 1,...,N$.
  }%
  \label{fig:pieceWiseLinear}%
\end{figure*}

A direct way to proceed in this case is to use a basis that 
incorporates the constraints:
\begin{equation*}
h_1(X)=1,~h_2(X)=X,~ h_3(X)=(X-\xi_1)_+, ~h_4(X)=(X-\xi_2)_+ 
\end{equation*}
\noindent where $t_+$ denotes the positive part.
Figure~\ref{fig:pieceWiseCubic} shows a series of piecewise-cubic 
polynomials fit to the same data, with increasing orders of 
continuity at the knots. The function in the lower right panel 
is continuous, and has continuous first and second 
derivatives at the knots. It is known as a cubic spline.
\begin{figure*}[ht!]
%  \centering
  \captionsetup{justification=centering}
  \includegraphics[width=0.6\textwidth]{00_figures/pieceWiseCubic}
  \caption[][-14\baselineskip]{A series of piecewise-cubic 
  polynomials, with increasing orders of continuity.
  }%
  \label{fig:pieceWiseCubic}%
\end{figure*}

The following basis represents a cubic spline with knots at $\xi_1$ 
and $\xi_2$:
\begin{equation}
\begin{aligned}\label{eq:cubicSpline}
h_1(X)=1,~h_3(X)=X^2,~h_5(X)=(X-\xi_1)_+,\\
~h_2(X)=X, h_4(X)=X^3, ~h_6(X)=(X-\xi_2)_+.
\end{aligned}
\end{equation}
More generally, an order-M spline with 
knots $\xi_j$, $j = 1, \dots,K$ is a piecewise-polynomial of order 
$M$, and has continuous derivatives up to order $M-2$. 
A cubic spline has $M = 4$\marginnote{I do not understand this. A cubic spline 
has to be a polynomial of degree 3 in each interval, is not it? Or, it seems cubic 
splice is a splice that its first and second derivatives at the knots are continuous. 
And the degree of polynomials in each interval is a function of
order of splice and the how much continuity is needed at the knots.}.
In fact the piecewise-constant 
function in~\cref{fig:pieceWiseLinear} is an order-1 
spline, while the continuous piece-wise linear 
function is an order-2 spline. Likewise the general 
form for the truncated-power basis set would be
\begin{equation}
\begin{aligned}\label{eq:cubicSpline}
h_j(X) &= X^{j-1}, ~~~~~~~~~~~~~~ j=1, \dots, M\\
h_{M+\ell}(X) &= (X-\xi_\ell)_+^{M-1}, ~~\ell=1, \dots, K.
\end{aligned}
\end{equation}



\subsection{Filtering and Feature Extraction}
\label{FilteringandFeatureExtraction}

\subsection{Smoothing Splines}
\label{SmoothingSplines}

\subsection{Automatic Selection of the Smoothing Parameters}
\label{AutomaticSelectionoftheSmoothingParameters}

\subsection{Nonparametric Logistic Regression}
\label{NonparametricLogisticRegression}

\subsection{Multidimensional Splines}
\label{MultidimensionalSplines}
