\chapter{Support Vector Machines}
\label{chap:Support_Vector_Machines}

\section{Introduction}
\label{sec:SVM_Introduction}

I start my notes with the Chapter 12 of Tibshirani's 
book\cite{Hastie2001elements} and then move to the 
Steinwart's book\cite{steinwart2008support}; Support Vector Machines.

Optimal separating hyperplanes are for the case when two classes are 
linearly separable; it provides a linear decision boundaries for classification. 
Here we cover extensions to the non-separable case, 
where the classes overlap. These techniques are then generalized to 
what is known as the support vector machine, which produces 
nonlinear boundaries by constructing a linear boundary in a large, 
transformed version of the feature space.
The second set of methods generalize Fisher's linear 
discriminant analysis (LDA). The generalizations include 
flexible discriminant analysis which facilitates construction of 
nonlinear boundaries in a manner very similar to the support 
vector machines, penalized discriminant analysis for 
problems such as signal and image classification 
where the large number of features are highly 
correlated, and mixture discriminant analysis for 
irregularly shaped classes.


