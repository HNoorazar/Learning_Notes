\section{Graphical Tools}
\label{sec:Graphical_Tools}
In this section we expand on graphical tools that
are briefly mentioned in~\cref{sec:SmoothinginDetail}.
Some examples that use residual plots for investigations
are given in~\cref{sec:problems_Interview}.
Graphical tools are not a way to compare models against each
other. It is rather a way of model assessment.

\subsection{Residual Plots}
\label{sec:Residual-Plots}
From\cite{loader2006local} we have:
The most important diagnostic component is the residuals. 
For local regression, the residuals are defined as the difference 
between observed and fitted values:
\begin{equation}
\hat \varepsilon_i = Y_i - \hat f(x_i).
\end{equation}

One can use the residuals to construct formal tests of 
goodness of fit or to modify the local regression estimate 
for nonhomogeneous variance. These topics will be explored 
more in Chapter 9. For practical purposes, most insight 
is often gained simply by plotting the residuals in 
various manners. Depending on the situation, plots 
that can be useful include:
\begin{enumerate}
\item Residuals vs. predictor variables, for detecting lack of fit, such as a trimmed peak.

\item Absolute residuals vs. the predictors, to detect 
dependence of residual variance on the predictor variables. 
One can also plot absolute residuals vs. 
fitted values, to detect dependence of the residual 
variance on the mean response.

\item Q-Q plots (Wilk and Gnanadesikan 1968), to detect 
departure from normality, such as skewness or heavy tails, 
in the residual distribution. If non-normality is found, fitting 
criteria other than least squares may produce better 
results. See Section 6.4.

\item Serial plots of $\hat \varepsilon_i $ vs. $\hat \varepsilon_{i-1} $, 
to detect correlation between residuals.

\item Sequential plot of residuals, in the order the 
data were collected. In an industrial experiment, 
this may detect a gradual shift in experimental 
conditions over time.
\end{enumerate}

It is important not to look at the residual plots alone, 
but to use them in conjunction with plots of the fit. 
The object is to determine whether large residuals 
correspond to features in the data that have been 
inadequately modeled. The purpose of the plots can 
be related to the bias-variance trade-off:
\begin{itemize}
\item Plots of the fit help us detect noise in the fit.
\item Residual plots help us detect bias.
\end{itemize}

